%%%%%%%%%%%%%%%%%%%%%%%%%%%%%%%%%%%%%%%%%%%%%%%%%%%%%%%%%%%%%%%%%%%%%%%%%
% GRAU PRETENDIDO E TIPO DE MONOGRAFIA									%
%%%%%%%%%%%%%%%%%%%%%%%%%%%%%%%%%%%%%%%%%%%%%%%%%%%%%%%%%%%%%%%%%%%%%%%%%
% Alguns exemplos seguem abaixo. Se o seu for algum deles, descomente-o. Em geral, o grau e o tipo de monografia associado estão na mesma linha.
%\grau{Título}{especificação} \tipodemonografia{"a" para feminino e "e" para masculino}{Tipo}
%\grau{Doutor}{em Engenharia Elétrica} \tipodemonografia{a}{Tese de Doutorado}
\grau{Mestre}{em Engenharia de Sistemas Eletrônicos e Automação} \tipodemonografia{a}{Dissertação de Mestrado}
%\grau{Engenheiro}{Eletricista} \tipodemonografia{o}{Trabalho de Conclusão de Curso}

%%%%%%%%%%%%%%%%%%%%%%%%%%%%%%%%%%%%%%%%%%%%%%%%%%%%%%%%%%%%%%%%%%%%%%%%%
% TÍTULO																%
%%%%%%%%%%%%%%%%%%%%%%%%%%%%%%%%%%%%%%%%%%%%%%%%%%%%%%%%%%%%%%%%%%%%%%%%%
% Os comandos a seguir servem para definir o título do trabalho. Para evitar  que o latex defina automaticamente a quebra de linha, foram definidos um comando por linha. Desta forma o autor define como quer que o título seja dividio em várias linhas. O exemplo abaixo é para um título que ocupa três linhas. Observe que mesmo com a linha 4 não sendo utilizada, o comando \titulolinhaiv é chamado.
\titulolinhai{TÍTULO DA TESE~}
\titulolinhaii{QUE PODE PULAR ATÉ~}
\titulolinhaiii{ALGUMAS LINHAS}
\titulolinhaiv{}

%%%%%%%%%%%%%%%%%%%%%%%%%%%%%%%%%%%%%%%%%%%%%%%%%%%%%%%%%%%%%%%%%%%%%%%%%
% AUTORES																%
%%%%%%%%%%%%%%%%%%%%%%%%%%%%%%%%%%%%%%%%%%%%%%%%%%%%%%%%%%%%%%%%%%%%%%%%%
% Os nomes dos autores são definidos pelos comandos \autori (autor 1) e \autorii (autor 2). Para trabalhos com apenas um autor, deve-se usar \autorii{} para que não apareça um nome para segundo autor.
\autori{Nome do Autor 1}
\autorii{Nome do Autor 2}
\autorii{} % descomente esta linha se não houver segundo autor.
\autoriii{Nome do Autor 3}
\autoriii{} % descomente esta linha se não houver terceiro autor.

%%%%%%%%%%%%%%%%%%%%%%%%%%%%%%%%%%%%%%%%%%%%%%%%%%%%%%%%%%%%%%%%%%%%%%%%%
% BANCA EXAMINADORA														%
%%%%%%%%%%%%%%%%%%%%%%%%%%%%%%%%%%%%%%%%%%%%%%%%%%%%%%%%%%%%%%%%%%%%%%%%%
% Os nomes dos membros da banca são definidos a seguir. Pode-se ter até 5 membros da banca, numerados de i a v (algarismos romanos).
% Para trabalhos com apenas um autor, deve-se usar \autorii{} para que não apareça um nome para segundo autor. É incubência do usuário definir no argumento dos comandos a afiliação do membro da banca, assim como sua posição (se for orientador ou co-orientador). 
% Os nomes definidos pelos comandos abaixo aparecem na ordem de i a v.
\membrodabancai{Prof. Nome do Orientador, Ph.D, FT/UnB}
\membrodabancaifuncao{Orientador}
\membrodabancaii{}
\membrodabancaiifuncao{}
\membrodabancaiii{}
\membrodabancaiiifuncao{Examinador interno}
\membrodabancaiv{}
\membrodabancaivfuncao{Examinador interno}
\membrodabancav{}
\membrodabancavfuncao{}

%%%%%%%%%%%%%%%%%%%%%%%%%%%%%%%%%%%%%%%%%%%%%%%%%%%%%%%%%%%%%%%%%%%%%%%%%
% DATA DA DEFESA														%
%%%%%%%%%%%%%%%%%%%%%%%%%%%%%%%%%%%%%%%%%%%%%%%%%%%%%%%%%%%%%%%%%%%%%%%%%
\mes{junho}
\ano{2015}

%%%%%%%%%%%%%%%%%%%%%%%%%%%%%%%%%%%%%%%%%%%%%%%%%%%%%%%%%%%%%%%%%%%%%%%%%
% FICHA CATALOGRÁFICA													%
%%%%%%%%%%%%%%%%%%%%%%%%%%%%%%%%%%%%%%%%%%%%%%%%%%%%%%%%%%%%%%%%%%%%%%%%%
%Colocar o nome do autor como vai aparecer no catálogo. Último sobrenome primeiro, depois o nome e sobrenomes intermediários. Ex.: Borges, Geovany Araújo
\autorcatalogo{Sobrenome, Nome}
%Colocar o nome abreviado. Último sobrenome primeiro, depois as iniciais do nome e sobrenomes intermediários. Ex.: Borges, G.A.
\autorabreviadocatalogo{Sobrenome, N.}

%%%%%%%%%%%%%%%%%%%%%%%%%%%%%%%%%%%%%%%%%%%%%%%%%%%%%%%%%%%%%%%%%%%%%%%%%
% PALAVRAS CHAVE														%
%%%%%%%%%%%%%%%%%%%%%%%%%%%%%%%%%%%%%%%%%%%%%%%%%%%%%%%%%%%%%%%%%%%%%%%%%
\palavraschavecatalogoi{Palavra chave 1}
\palavraschavecatalogoii{Palavra chave 2}
\palavraschavecatalogoiii{Palavra chave 3}
\palavraschavecatalogoiv{Palavra chave 4}

%%%%%%%%%%%%%%%%%%%%%%%%%%%%%%%%%%%%%%%%%%%%%%%%%%%%%%%%%%%%%%%%%%%%%%%%%
% NÚMERO DA PUBLICAÇÃO													%
%%%%%%%%%%%%%%%%%%%%%%%%%%%%%%%%%%%%%%%%%%%%%%%%%%%%%%%%%%%%%%%%%%%%%%%%%
%fornecido pelo departamento após a defesa
%\publicacao{TCC-}

%Número de páginas da dissertação.
\numeropaginascatalogo{\pageref{LastPage}~p.}